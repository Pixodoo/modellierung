\documentclass{article}
\usepackage[utf8]{inputenc}
\usepackage{graphicx}
\graphicspath{ {./images/} }
\usepackage{tikz}
\usepackage[normalem]{ulem} 

\begin{document}

\section{SQL-Anfragen}
\subsection{(a)}
SELECT Vorname, Name, Wohnort\\
FROM Angestellter\\
WHERE Wohnort = 'Leipzig'\\
ORDER BY Geburtstag\\

\subsection{(b)}
SELECT Wohnort, Projektort, Ort
FROM Angestellter, Projekt, Abteilung\\
WHERE Projekt.Projektort = Angestellter.Wohnort
AND Abteilung.AbteilungsNr = Angestellter.AbteilungsNr \\
 
 
\subsection{(c)}
SELECT DISTINCT Beruf, COUNT$(*)_{>3}$ AS Anzahl\\
FROM Angestellter\\
GROUP BY Beruf\\


\subsection{(d)}
SELECT DISTICT Beruf, \\
FROM Angestellter \\
WHERE 




\section{SQL-Anweisungen}
\subsection{(a)}
INSERT INFO Angestellter 
SET PersonalNr=101, Name='Kraft', Vorname='Simone', Beruf='Physikerin', Gehalt='65000', AbteilungsNr=2,
Manager=NULL, Geburtstag=NULL, Wohnort=NULL

\subsection{(b)}
DELETE FROM Abteilung\\
WHERE Ort='Berlin'\\

\subsection{(c)}
Update Angestellter\\
SET AbteilungsNr=5\\
WHERE AbteilungsNr IS NULL\\

\subsection{(d)}



\subsection{(e)}
TRUNCATE Abteilung

\subsection{(f)}

    
\end{document}