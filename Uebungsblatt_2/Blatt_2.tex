\documentclass{article}
\usepackage[utf8]{inputenc}
\usepackage{graphicx}
\graphicspath{ {./images/} }
\usepackage{tikz}
\usepackage[normalem]{ulem} 

\title{Übungsblatt 2}
\author{Robin Jahn 3598632 st178507@stud.uni-
stuttgart.de \\
Alexander Denk ... }

\begin{document}

\section{Umsetzung eines ER-Diagramms ins Relationenmodell}

\subsection{(a)}
\textbf{AUFENTHALT}(\dashuline{von}, bis, von) \\
\textbf{GAST}(\underline{KundenNr}, Name, Adresse(Strasse, Ort)) \\ 
\textbf{RECHNUNG}(\underline{ReNr}, Datum, Summe, zu)\\
\textbf{ZIMMER}(\underline{Nummer}, Betten, in)\\
\textbf{KATEGORIE}(\underline{Bezeichnung}, {Ausstattung}, hat)\\
\textbf{RESTAURANT}(\underline{ID}, RestaurantName, Plätze)\\
\textbf{GETRÄNK}(\underline{ProdNr}, Preis, ProduktName, Größe)\\
\textbf{SPEISE}(\underline{ProdNr}, Preis, ProduktName, Gewicht)\\
\textbf{PRODUKT}(\underline{ProdNr}, Preis, ProduktName)\\ \\
\textbf{RESTORAUNTBESUCH}(\dashulin{von}, \underline{ID}, \underline{ProdNr}, Datum, Tisch)\\
\textbf{MINIBARKONSUM}(\dashulin{von}, \underline{ProdNr}, Anzahl)\\    
    

\subsection{(b)}
\textbf{AUFENTHALT}(von) -> \textbf{GAST}(KundenNr)
\textbf{RECHNUNG}(zu) -> \textbf{AUFENTHALT}(von)
\textbf{ZIMMER}(in) -> \textbf{AUFENTHALT}(von)
\textbf{KATEGORIE}(hat) -> \textbf{ZIMMER}(Nummer)
\textbf{AUSSTATTUNG}(Bezeichnung) -> \textbf{KATEGORIE}(Bezeichnung)
\textbf{RESTORAUNTBESUCH}(von) -> \textbf{AUFENTHALT}(von)
\textbf{RESTORAUNTBESUCH}(ID) -> \textbf{RESTAURANT}(ID)
\textbf{RESTORAUNTBESUCH}(ProdNr) -> \textbf{PRODUKT}(ProdNr)
\textbf{MINIBARKONSUM}(von) -> \textbf{AUFENTHALT}(von)
\textbf{MINIBARKONSUM}(ProdNr) -> \textbf{PRODUKT}(ProdNr)

\subsection{(c)}
\textbf{Partitionierungsmodell:} \\
\textbf{GETRÄNK} (ProdNr, Größe)
\textbf{SPEISE}(ProdNr, Gewicht)\\ \\

\textbf{Weitere FK für Partitionierungsmodell:}\\
\textbf{GETRÄNK}(ProdNr) -> PRODUKT(ProdNr)
\textbf{SPEISE}(ProdNr) -> PRODUKT(ProdNr)\\

Beim Hauskastenmodell ist eine \textbf{volle Redundanz} vorhanden, wobei hier das Partitionierungsmodell
nur den Schlüsselwert von Produkt und die eigenen Attribute darstellt

\section{Anfragen in der Relationenalgebra}
\subsection{(a)}
Geben Sie Namen und Raum aller Professoren vom Rang ‚C4‘ aus\\
$\Pi_{Name} (\Pi_{Raum}(\sigma_{Rang='C4'} Professoren))$ \\

\subsection{(b)}
Geben Sie die Namen aller Hörer der Vorlesung ‚Logik‘ aus
$\Pi_{Name}((Student) \bowtie horen \bowtie (\sigma_{Titel='Logik'}Vorlesung))$ \\

\subsection{(c)}
Erstellen Sie eine Liste die die Name n aller Assistenten und jeweils den
Namen des zugehörigen Vorgesetzten(‘Boss‘) enthält. Formulieren Sie diese Anfrage einmal mit und einmal 
ohne Join-Operator\\


\subsection{(d)}
Erstellen Sie eine Liste aller Namen. Es sollen Namen von Professoren,
Studenten und Assistenten berücksichtigt werden. \\
$((\Pi_{Name} Professoren) \cup (\Pi_{Name} Studenten) \cup (\Pi_{Name} Assistenten))$

\subsection{(e)}
Welche Studenten (MatrNr) haben noch an keiner Prüfung teilgenommen?\\
$((\Pi_{MatrNr} Studenten) - (\Pi_{MatrNr} pruefen))$\\

\subsection{(f)}
Welches ist die durchschnittliche Semesteranzahl aller Studenten?\\
$(\textbf{Sum}(\Pi_{Semester} Studenten)) \div  (\textbf{Count}(Studenten))$\\

\subsection{(g)}
Wieviele Hörer gibt es pro Vorlesung? Erstellen Sie eine Liste pro Vorle-
sung (VorlNr).\\
$ $

\end{document}