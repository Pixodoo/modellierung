\documentclass{article}
\usepackage[utf8]{inputenc}
\usepackage{graphicx}
\graphicspath{ {./images/} }
\usepackage{tikz}
\usepackage[normalem]{ulem} 

\title{Übungsblatt 2}
\author{Robin Jahn 3598632 st178507@stud.uni-
stuttgart.de \\
Alex Denk ... }

\begin{document}

\section{Umsetzung eines ER-Diagramms ins Relationenmodell}

\subsection{(a)}
\textbf{AUFENTHALT}(\dashulin{von}, bis, von) \\
\textbf{GAST}(\underline{KundenNr}, Name, Adresse(Strasse, Ort)) \\ 
\textbf{RECHNUNG}(\underline{ReNr}, Datum, Summe, zu)\\
\textbf{ZIMMER}(\underline{Nummer}, Betten, in)\\
\textbf{KATEGORIE}(\underline{Bezeichnung}, {Ausstattung}, hat)\\
\textbf{RESTAURANT}(\underline{ID}, RestaurantName, Plätze)\\
\textbf{GETRÄNK}(\underline{ProdNr}, Preis, ProduktName, Größe)\\
\textbf{SPEISE}(\underline{ProdNr}, Preis, ProduktName, Gewicht)\\
\textbf{PRODUKT}(\underline{ProdNr}, Preis, ProduktName)\\ \\
\textbf{RESTORAUNTBESUCH}(\dashulin{von}, \underline{ID}, \underline{ProdNr}, Datum, Tisch)\\
\textbf{MINIBARKONSUM}(\dashulin{von}, \underline{ProdNr}, Anzahl)\\    
    

\subsection{(b)}



\end{document}