\documentclass{article}
\usepackage[utf8]{inputenc}
\usepackage{graphicx}
\graphicspath{ {./images/} }
\usepackage{tikz}
\usepackage[normalem]{ulem} 

\begin{document}
\section{SQL und Relationenalgebra}
\subsection{(a)}
$\Pi_{k \underline{ }name}(\sigma_{(k \underline{ }alter)> 40} mitarbeiter)$

\subsection{(b)}
$\Pi_{b\underline{ }ID}(\sigma_{(d\underline{ }kredithoehe)>20000} (darlehen \bowtie (\sigma_{a\underline{ }ort='München'}(mitarbeiter \bowtie adresse \bowtie bankkonto)))) $


\subsection{(c)}
SELECT k_name, k_alter, a_plz \\
FROM kunde NATURLA JOIN adresse \\
ORDER BY k_alter DESC \\

\subsection{(d)}
SELECT m_name, anzahl_an_bankkonten \\
FROM .... \\
WHERE m_alter\geq 30 AND   \\\

(SELECT COUNT(*)AS anzahl_an_bankkonten\\
FROM mitarbiter NATRUAL JOIN bankkonto\\
GROUP BY m_id\\)

 \subsection{(e)}
 SELECT DISTINCT k_name, COUNT(.....) AS kredit_anzahl\\
 FROM kunde NATRUAL JOIN darlehen\\
 WHERE COUNT(d_kunde = k_id)>1 AND ...
 %Durchschnittliche Kredithöhe 
 (d_Kredithoehe \geq (SUM d_Kredithohe) \div (COUNT d_Kredithohe))

 \subsection{(f)}


\end{document}